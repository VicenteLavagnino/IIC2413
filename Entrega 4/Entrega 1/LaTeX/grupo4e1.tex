\documentclass[12pt]{article}
\usepackage[spanish]{babel}
\usepackage[utf8]{inputenc}
\usepackage{fullpage,graphicx,amssymb,amsmath,algorithmic}
\usepackage{pgfplots}
\usepackage{amsmath}
\pgfplotsset{compat = newest}

\begin{document}
\thispagestyle{empty}
\noindent Pontificia Universidad Católica de Chile\\
\noindent Departamento de Ciencia de la Computación\\
\noindent IIC2413 - Bases de Datos\\
\begin{center}
{\Huge\bf Entrega 1}\\
\vspace{1em}
\footnotesize{Vicente Lavagnino e Ignacio Laval}\\
\vspace{1em}
\rule{\textwidth}{0.1mm}
\end{center}
\thispagestyle{plain}

\section{Tablas presentes en la base de datos}

\noindent Videojuegos(\underline{id}, titulo, clasificacion, fecha\_lanzamiento, precio, puntuacion)
\begin{table}[h]
    \centering
    \begin{tabular}{|l|l|l|l|l|l|l|}
    \hline
    \underline{id} & Titulo & Clasificacion & Fecha de Lanzamiento & Precio & Puntuacion \\ \hline
       &        &           &                 &        &           \\ \hline
    \end{tabular}
\end{table}

\noindent Generos(\underline{id}, nombre)
\begin{table}[h]
    \centering
    \begin{tabular}{|l|l|}
    \hline
    \underline{id} & Nombre \\ \hline
       &        \\ \hline
    \end{tabular}
\end{table}

\noindent Subgeneros(\underline{id}, nombre)
\begin{table}[h]
    \centering
    \begin{tabular}{|l|l|}
    \hline
    \underline{id} & Nombre \\ \hline
       &        \\ \hline
    \end{tabular}
\end{table}

\noindent Proveedores(\underline{id}, nombre, plataforma)
\begin{table}[ht!]
    \centering
    \begin{tabular}{|l|l|l|}
    \hline
    \underline{id} & Nombre & Plataforma \\ \hline
       &        &            \\ \hline
    \end{tabular}
\end{table}

\noindent Usuarios(\underline{id}, nombre, email, contraseña)
\begin{table}[ht!]
    \centering
    \begin{tabular}{|l|l|l|l|}
    \hline
    \underline{id} & Nombre & Email & Contraseña \\ \hline
       &        &       &                 \\ \hline
    \end{tabular}
\end{table}
\newpage
\noindent Reseñas(\underline{id\_usuario}, \underline{id\_videojuego}, texto, veredicto)
\begin{table}[ht!]
    \centering
    \begin{tabular}{|l|l|l|l|}
    \hline
    \underline{id\_usuario} & \underline{id\_videojuego} & Texto & Veredicto \\ \hline
       &        &       &                 \\ \hline
    \end{tabular}
\end{table}

\noindent PropiedadVideojuegos(\underline{id\_videojuego}, \underline{id\_usuario}, \underline{id\_proveedor}, registro\_horas, preorden)
\begin{table}[ht!]
    \centering
    \begin{tabular}{|l|l|l|l|l|}
    \hline
    \underline{id\_videojuego} & \underline{id\_usuario} & \underline{id\_proveedor} & Registro de Horas & Preorden \\ \hline
       &        &            &       &           \\ \hline
    \end{tabular}
\end{table}

\noindent Suscripciones(\underline{id\_cuenta}, activo)
\begin{table}[ht!]
    \centering
    \begin{tabular}{|l|l|l|}
    \hline
    \underline{id\_usuario} & \underline{id\_proveedor} & Activo \\ \hline
       &        &            \\ \hline
    \end{tabular}
\end{table}

\noindent MercadoVideojuegos(\underline{id\_videojuego}, \underline{id\_proveedor}, precio)
\begin{table}[ht!]
    \centering
    \begin{tabular}{|l|l|l|}
    \hline
    \underline{id\_videojuego} & \underline{id\_proveedor} & Precio \\ \hline
       &        &            \\ \hline
    \end{tabular}
\end{table}

\noindent GeneroAsignados(\underline{id\_videojuego}, \underline{id\_genero})
\begin{table}[ht!]
    \centering
    \begin{tabular}{|l|l|}
    \hline
    \underline{id\_videojuego} & \underline{id\_genero} \\ \hline
       &        \\ \hline
    \end{tabular}
\end{table}

\noindent SubGeneroAsignados(\underline{id\_videojuego}, \underline{id\_subgenero})
\begin{table}[ht!]
    \centering
    \begin{tabular}{|l|l|}
    \hline
    \underline{id\_videojuego} & \underline{id\_subgenero} \\ \hline
       &        \\ \hline
    \end{tabular}
\end{table}

\noindent GenerosYSubgeneros(\underline{id\_genero}, \underline{id\_subgenero})
\begin{table}[ht!]
    \centering
    \begin{tabular}{|l|l|}
    \hline
    \underline{id\_genero} & \underline{id\_subgenero} \\ \hline
       &        \\ \hline
    \end{tabular}
\end{table}

\noindent VideojuegosRecomendados(\underline{id\_videojuego}, cantidad\_recomendaciones)
\begin{table}[ht!]
    \centering
    \begin{tabular}{|l|l|}
    \hline
    \underline{id\_videojuego} & Cantidad de Recomendaciones \\ \hline
       &        \\ \hline
    \end{tabular}
\end{table}

\newpage
\section{Consultas en la base de datos}
\subsection{Consultas en algebra relacional}
\begin{enumerate}
    \item
    $\pi_{\text{videojuegos.titulo}, \text{proveedores.nombre}}$\\
    $\left( \left( \text{videojuegos} \bowtie_{\text{videojuegos.id} = \text{mercado\_videojuegos.id\_videojuego}} \text{mercado\_videojuegos} \right) \right.$\\
    $\left. \bowtie_{\text{proveedores.id} = \text{mercado\_videojuegos.id\_proveedor}} \text{proveedores} \right)$
    \item $
    \pi_{\text{titulo}} 
    \left( 
        \sigma_{\text{cantidad\_de\_recomendaciones} \geq 5} 
        \left( 
            \text{videojuegos} \bowtie_{\text{id} = \text{id\_videojuego}} \text{videojuegos\_recomendados} 
        \right) 
    \right)
    $
    \item $\pi_{titulos}
    \left(
        videojuegos \bowtie_{id\_videojuego=id\_videojuego} 
        \left( 
            genero\_asignado \bowtie_{id\_genero=id} 
            \sigma_{nombre='fps'}(generos) 
        \right) 
    \right) 
$\\$\bigcup \pi_{titulos}
\left(
    videojuegos \bowtie_{id\_videojuego=id\_videojuego} 
    \left( 
        subgenero\_asignado \bowtie_{id\_subgenero=id\_subgenero} 
    \right)
\right.$ \\ $
\left.
    \left( 
        relacion\_subgenero \bowtie_{id\_genero=id} 
        \sigma_{nombre='fps'}(generos) 
    \right) 
\right)
$
\end{enumerate}
\subsection{Consultas en SQL}
\begin{enumerate}
    \item SELECT videojuegos.titulo, proveedores.nombre FROM videojuegos JOIN mercado\_videojuegos ON videojuegos.id = mercado\_videojuegos.id\_videojuego JOIN proveedores ON \mbox{proveedores.id} = mercado\_videojuegos.id\_proveedor
    \item SELECT videojuegos.titulo FROM videojuegos JOIN videojuegos\_recomendados ON videojuegos.id = videojuegos\_recomendados.id\_videojuego WHERE \\videojuegos\_recomendados.cantidad\_de\_recomendaciones $>=$ 5;
    \item SELECT videojuegos.titulo FROM videojuegos JOIN genero\_asignado ON videojuegos.id = genero\_asignado.id\_videojuego WHERE genero\_asignado.id\_genero = \mbox{(SELECT} id FROM generos WHERE nombre = 'fps') UNION SELECT videojuegos.titulo FROM videojuegos JOIN subgenero\_asignado ON videojuegos.id = subgenero\_asignado.id\_videojuego WHERE subgenero\_asignado.id\_subgenero IN (SELECT id\_subgenero FROM relacion\_subgenero WHERE id\_genero = (SELECT id FROM generos WHERE nombre = 'fps'));
\end{enumerate}



\end{document}
